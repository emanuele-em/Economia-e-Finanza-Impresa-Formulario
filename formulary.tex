\documentclass[fontsize=8pt]{scrartcl}
\usepackage[T1]{fontenc}
\usepackage[utf8]{inputenc}
\usepackage{multicol} % for multicols environment
\usepackage{mathtools} % loads amsmath, for math environments etc
\usepackage{amsmath}
\usepackage{geometry} % for defining margins etc
\usepackage{tabularx}% http://ctan.org/pkg/tabularx
\usepackage{makecell}
\usepackage{cancel}
\usepackage{float}
\usepackage{xcolor}
\usepackage{amsmath}
\usepackage{tikz}
\usetikzlibrary{calc}
\renewcommand\tabularxcolumn[1]{m{#1}}
\geometry{
 margin=0.7cm
}

\tikzset{
  % Styling of header text is done using key/value options for TikZ nodes. See
  % section 16.4 of the PGF manual for a complete list of options that affect
  % text.
  headings/base/.style = {
    % Zap node seperation, set text width and alignment.
    outer sep = 0pt,
    % Trim off 2/3rd of an em to compensate for the inner xsep which spaces the
    % text nicely away from the left side, but causes the node to hang into the
    % right margin.
    text width = {\textwidth - 0.6666em},
    align = left,
    text = white,
  },
  headings/section/.style = {
    headings/base,
    fill = black!100,
    font = \sffamily\huge
  },
  headings/subsection/.style = {
    headings/base,
    fill = black!75,
    font = \sffamily\LARGE
  },
  headings/subsubsection/.style = {
    headings/base,
    fill = black!25,
    text=black!75,
    font = \sffamily\large
  }
}

\newcommand{\colorboxedsec}[2]{%
  \tikz{\node[headings/#1]{#2};}}

\setkomafont{section}{\colorboxedsec{section}}
\setkomafont{subsection}{\colorboxedsec{subsection}}
\setkomafont{subsubsection}{\colorboxedsec{subsubsection}}

\newenvironment{definitions*}
  {\par\vspace{\abovedisplayskip}\noindent
   \tabularx{\columnwidth}{>{$}l<{$} @{${}={}$} >{\raggedright\arraybackslash}X}}
  {\endtabularx\par\vspace{\belowdisplayskip}}

\definecolor{light-gray}{gray}{0.95}
% put color to \boxed math command
\newcommand*{\boxcolor}{light-gray}
\makeatletter
\renewcommand{\boxed}[1]{
    \textcolor{\boxcolor}{%
        \tikz[baseline={([yshift=-1ex]current bounding box.center)}]
        \node [fill=light-gray, minimum width=1ex,draw]
        {\normalcolor\m@th$\displaystyle#1$};
    }
}
 \makeatother
\allowdisplaybreaks % allow environments like gather and align to break across columns/pages
\begin{document}

\section{TEORIA DELL'IMPRESA}

\subsection{Capitalismo manageriale}

\subsubsection{Novitá rispetto alla teoria neoclassica d'impresa}

\subsubsection{Separazione tra proprietá e controllo}

\subsubsection{Modello di R. Marris}

\begin{definitions*}
    g_d & tasso di crescita della domanda\\
    d_s & tasso di diversificazione con successo\\
    P & tasso di profitto\\
    \pi & profitto\\
    K & Capitale investito\\
    g_d & $ f(d_s) $\\
    d_s & $ f(P) $\\
    P & $ \frac{\pi}{K} $\\
    P & $ f(g_d) $\\
    g_s & tasso di crescita dell'offerta = $ \frac{\Delta K}{K} = \frac{I}{K} $\\
    \rho & tasso di reinvestimento\\
    \epsilon & tasso di aumento del debito\\
    I & investimenti = impieghi\\
    \rho\pi & fonti interne, profitto reinvestiti \\
    \epsilon I & fonti esterne, aumenti di capitale di debito
\end{definitions*}

\begin{align*}
    I &= \rho\pi + \epsilon I \\
    \pi &= \left[ \frac{1-\epsilon}{\rho} \right] I \\
    P &= \frac{\pi}{K} = \underbrace{\frac{1-\epsilon}{\rho}}_{\beta}\underbrace{I \frac{1}{K}}_{g_s} = \beta g_s
\end{align*}

\subsubsection{La funzione di utilitá del manager}

\begin{align*}
    V &= \text{Tasso di valutazione / valuation ratio} = \frac{V_m}{K}\\
    V_m &= \text{Valore di mercato} = \sum_{t=0}^{+\infty}(1-\rho)\phi_0\left[\frac{(1+g)^t}{(1+i)^t}\right] \overset{\text{converge a}}{\longrightarrow} (1-\rho)\pi_0\left[\frac{1+i}{1-g}\right]\\
    \pi_0 &= \text{profitto al tempo } t_0\\
    g &= \text{tasso di crescita}\\
    i &= \text{tasso di interesse}\\
    V &= \frac{V_m}{K} = \left(\frac{\pi_0}{K} - \frac{\rho \pi_0}{K}\right)\left[\frac{1+i}{1-g}\right]\\
\end{align*}
dato che \(P=\frac{\pi}{K}\) e \(\rho\pi = I\) allora:
\begin{align*}
    \frac{\rho \pi}{K} = \frac{I}{K} = g
\end{align*}
si ottiene
\begin{align*}
    V = \left[P(g)-g\right]\frac{1+i}{1-g}
\end{align*}
per capire l'andamento di \(V\) rispetto a \(g\) e \(P(g)\) si deriva:
\begin{align*}
    \frac{\partial V}{\partial g} = &\left[\frac{1+i}{(1-g)^2}\right]\left[(1-g)\frac{\partial P}{\partial g} + P-i\right]\\
    \text{se } &\frac{\partial P}{\partial g}>0 \rightarrow \frac{\partial V}{\partial g} > 0\\
    \text{se } &\frac{\partial P}{\partial g}<0 \rightarrow \text{prima }\frac{\partial V}{\partial g} > 0 \text{ poi }\frac{\partial V}{\partial g} = 0 \text{ poi }\frac{\partial V}{\partial g} < 0\\
\end{align*}

\section{TEORIE FINANZIARIE}

\section{CORPORATE GOVERNANCE}

\section{GOVERNANCE}

\end{document}