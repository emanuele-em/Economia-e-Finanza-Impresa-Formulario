\documentclass[fontsize=8pt]{scrartcl}
\usepackage[T1]{fontenc}
\usepackage[utf8]{inputenc}
\usepackage{multicol} % for multicols environment
\usepackage{mathtools} % loads amsmath, for math environments etc
\usepackage{amsmath}
\usepackage{geometry} % for defining margins etc
\usepackage{tabularx}% http://ctan.org/pkg/tabularx
\usepackage{makecell}
\usepackage{cancel}
\usepackage{float}
\usepackage{xcolor}
\usepackage{amsmath}
\usepackage{tikz}
\usetikzlibrary{calc}
\renewcommand\tabularxcolumn[1]{m{#1}}
\geometry{
 margin=0.7cm
}

\tikzset{
  % Styling of header text is done using key/value options for TikZ nodes. See
  % section 16.4 of the PGF manual for a complete list of options that affect
  % text.
  headings/base/.style = {
    % Zap node seperation, set text width and alignment.
    outer sep = 0pt,
    % Trim off 2/3rd of an em to compensate for the inner xsep which spaces the
    % text nicely away from the left side, but causes the node to hang into the
    % right margin.
    text width = {\textwidth - 0.6666em},
    align = left,
    text = white,
  },
  headings/section/.style = {
    headings/base,
    fill = black!100,
    font = \sffamily\huge
  },
  headings/subsection/.style = {
    headings/base,
    fill = black!75,
    font = \sffamily\LARGE
  },
  headings/subsubsection/.style = {
    headings/base,
    fill = black!25,
    text=black!75,
    font = \sffamily\large
  }
}

\newcommand{\colorboxedsec}[2]{%
  \tikz{\node[headings/#1]{#2};}}

\setkomafont{section}{\colorboxedsec{section}}
\setkomafont{subsection}{\colorboxedsec{subsection}}
\setkomafont{subsubsection}{\colorboxedsec{subsubsection}}

\newenvironment{definitions*}
  {\par\vspace{\abovedisplayskip}\noindent
   \tabularx{\columnwidth}{>{$}l<{$} @{${}={}$} >{\raggedright\arraybackslash}X}}
  {\endtabularx\par\vspace{\belowdisplayskip}}

\definecolor{light-gray}{gray}{0.95}
% put color to \boxed math command
\newcommand*{\boxcolor}{light-gray}
\makeatletter
\renewcommand{\boxed}[1]{
    \textcolor{\boxcolor}{%
        \tikz[baseline={([yshift=-1ex]current bounding box.center)}]
        \node [fill=light-gray, minimum width=1ex,draw]
        {\normalcolor\m@th$\displaystyle#1$};
    }
}
 \makeatother
\allowdisplaybreaks % allow environments like gather and align to break across columns/pages
\begin{document}

\section{TEORIA DELL'IMPRESA}

\subsection{Capitalismo manageriale}

  \subsubsection{Novitá rispetto alla teoria neoclassica d'impresa}

  \subsubsection{Separazione tra proprietá e controllo}

  \subsubsection{Modello di R. Marris}

  \begin{definitions*}
      g_d & tasso di crescita della domanda\\
      d_s & tasso di diversificazione con successo\\
      P & tasso di profitto\\
      \pi & profitto\\
      K & Capitale investito\\
      g_d & $ f(d_s) $\\
      d_s & $ f(P) $\\
      P & $ \frac{\pi}{K} $\\
      P & $ f(g_d) $\\
      g_s & tasso di crescita dell'offerta = $ \frac{\Delta K}{K} = \frac{I}{K} $\\
      \rho & tasso di reinvestimento\\
      \epsilon & tasso di aumento del debito\\
      I & investimenti = impieghi\\
      \rho\pi & fonti interne, profitto reinvestiti \\
      \epsilon I & fonti esterne, aumenti di capitale di debito
  \end{definitions*}

  \begin{align*}
      I &= \rho\pi + \epsilon I \\
      \pi &= \left[ \frac{1-\epsilon}{\rho} \right] I \\
      P &= \frac{\pi}{K} = \underbrace{\frac{1-\epsilon}{\rho}}_{\beta}\underbrace{I \frac{1}{K}}_{g_s} = \beta g_s
  \end{align*}

  \subsubsection{La funzione di utilitá del manager}

  \begin{definitions*}
      V & Tasso di valutazione / valuation ratio\\
      V_m & Valore di mercato\\
      \pi_0 & profitto al tempo \(t_0\)\\
      g & tasso di crescita\\
      i & tasso di interesse
  \end{definitions*}

  \begin{align*}
    V_m &= \sum_{t=0}^{+\infty}(1-\rho)\phi_0\left[\frac{(1+g)^t}{(1+i)^t}\right] \overset{\text{converge a}}{\longrightarrow} (1-\rho)\pi_0\left[\frac{1+i}{1-g}\right]\\
    V &= \frac{V_m}{K} = \left(\frac{\pi_0}{K} - \frac{\rho \pi_0}{K}\right)\left[\frac{1+i}{1-g}\right]\\
  \end{align*}
  dato che \(P=\frac{\pi}{K}\) e \(\rho\pi = I\) allora:
  \begin{align*}
      \frac{\rho \pi}{K} = \frac{I}{K} = g
  \end{align*}
  si ottiene
  \begin{align*}
      V = \left[P(g)-g\right]\frac{1+i}{1-g}
  \end{align*}
  per capire l'andamento di \(V\) rispetto a \(g\) e \(P(g)\) si deriva:
  \begin{align*}
      \frac{\partial V}{\partial g} = &\left[\frac{1+i}{(1-g)^2}\right]\left[(1-g)\frac{\partial P}{\partial g} + P-i\right] 
      &\begin{cases}
        \text{se } \frac{\partial P}{\partial g}>0 \rightarrow \frac{\partial V}{\partial g} > 0\\
          \text{se } \frac{\partial P}{\partial g}<0 \rightarrow \text{prima }\frac{\partial V}{\partial g} > 0 \text{ poi }\frac{\partial V}{\partial g} = 0 \text{ poi }\frac{\partial V}{\partial g} < 0\\
      \end{cases}
  \end{align*}
  


\subsection{Approccio Principale-Agente}

\subsubsection{La funzione di produzione di squadra}

\begin{definitions*}
  Q & funzione di produzione di squadra = \(f(x_1,x_2)\)\\
  \omega_1 & salario
\end{definitions*}
condizioni:
\begin{itemize}
  \item non separabili
  \item \(\frac{\partial^2Q}{\partial x_1\partial x_2}\)
\end{itemize}

\subsubsection{Shirking e Free riding}

\subsubsection{La visione contrattuale dell'impresa}

\subsubsection{Vantaggi e costi di squadra}

\begin{definitions*}
  B & Benefici totali\\
  T_e & Beneficio della squadra nel caso ottimale\\
  \frac{T_e}{2} & Beneficio della squadra nel caso reale\\
  C & Costi totali\\
  C_i & Costi del singolo\\  
  \pi & Profitto totale\\
  \pi_i & Profitto del singolo\\ 
  \frac{\partial\pi(e)}{\partial e} = 0 & Profitto massimizzato\\
  e & sforzo totale\\
  e_i & sforzo del singolo
\end{definitions*}
\textbf{Caso della ditta individuale - \(e^*\) sforzo ottimo}
\begin{align*}
  B &= b(e)\\
  C &= c(e)\\
  \pi(e) &= b(e)-c(e)\\
  \frac{\partial\pi(e)}{\partial e} &= \frac{\partial b(e)}{\partial e}-\frac{\partial c(e)}{\partial e} = 0\\
  \frac{\partial b(e)}{\partial e} &= \frac{\partial c(e)}{\partial d(e)}
\end{align*}
\textbf{Caso societá - \(e_i^* < e*\) sforzo del singolo minore di quello ottimo}
\begin{align*}
  B &= b(e_1) + b(e_2)\\
  C_i &= c(e_1)\\
  \pi(e_i) &=\frac{b(e_1) + b(e_2)}{2}-c(e_i)  
\end{align*}
Si deriva per trovare gli sforzi ottimali:
\begin{align*}
  \frac{\partial \pi_1}{\partial e_1} &= \frac{1}{2}\frac{\partial b(e_1)}{\partial e_1}-\frac{\partial c(e_1)}{\partial e_1} = 0\\
  \frac{\partial \pi_2}{\partial e_2} &= \frac{1}{2}\frac{\partial b(e_2)}{\partial e_2}-\frac{\partial c(e_2)}{\partial e_2} = 0\\
  e^p_1 &= \text{sforzo che massimizza il beneficio di 1}\\
  e^p_2 &= \text{sforzo che massimizza il beneficio di 2}\\
\end{align*}
\textbf{Caso Squadra senza Free Riding - \(e_i^{T^*} > e*\) sforzo congiunto della squadra otitmo maggiore di quello ottimo}
\begin{align*}
  B &= T(e_1 + e_2) = T_e > b(e_1) + b(e_2)\\
  C_i &= c(e_1)\\
  \pi &=T(e_1 + e_2) - c(e_1) - c(e_2)
\end{align*}
Si deriva per trovare gli sforzi ottimali:
\begin{align*}
  \frac{\partial \pi}{\partial e_1} &= \frac{\partial T}{\partial e_1} - \frac{\partial c(e_1)}{\partial e_1}\\
  \frac{\partial \pi}{\partial e_2} &= \frac{\partial T}{\partial e_2} - \frac{\partial c(e_2)}{\partial e_2}\\
  e^{T*}_1 &= \text{sforzo che massimizza il beneficio di 1}\\
  e^{T*}_2 &= \text{sforzo che massimizza il beneficio di 2}\\
\end{align*}
\textbf{Caso Squadra con Free Riding - \(e_i^T < e_i^{T^*}\) sforzo congiunto della squadra minore di quello congiunto della squadra ottimo}
\begin{align*}
  B &= T(e_1 + e_2) = T_e > b(e_1) + b(e_2)\\
  C_i &= c(e_1)\\
  \pi_i &=\frac{T(e_1 + e_2)}{2} - c(e_1)\\
\end{align*}
Si deriva per trovare gli sforzi ottimali:
\begin{align*}
  \frac{\partial \pi}{\partial e_1} &= \frac{1}{2}\frac{\partial T}{\partial e_1} - \frac{\partial c(e_1)}{\partial e_1}\\
  \frac{\partial \pi}{\partial e_2} &= \frac{1}{2}\frac{\partial T}{\partial e_2} - \frac{\partial c(e_2)}{\partial e_2}\\
  e^{T*}_1 &= sforzo che massimizza il beneficio di 1\\
  e^{T*}_2 &= sforzo che massimizza il beneficio di 2\\
\end{align*}

\subsubsection{Approccio principale - agente}

\subsubsection{Assicuratore e assicurato}

\subsubsection{Avversione al rischio e vincoli di agente}

\subsubsection{Contratto ottimo di incentivo}

\begin{definitions*}
  u & Utilità del manager = \(\sqrt{y} - (e-1)\)\\
  y & Reddito del manager\\
  e & Sforzo del manager\\
  \widehat{u} & Utilità di riserva (y=0, e=0)\\
  e^H & sforzo del manager alto = 2\\
  e^L & sforzo del manager basso = 1\\
  \epsilon & casualitá \\
  G & casualitá favoervole\\
  B & casualitá Sfavorevole\\
  p_H & Probabilitá di successo se lo sforzo è H\\
  p_L & Probabilitá di successo se lo sforzo è L\\
  \pi^B & profitto in caso di B = 6\\
  \pi^G & Profitto in caso di G = 36\\
\end{definitions*}

\textbf{Caso di piena informazione - First Best}

l'impresa vale \(e^H\) e lo garantisce con uno stipendio \(y^H\) (si suppone \(e^H=2\), \(e^L=1\))
\begin{align*}
  u = \sqrt{y^H} - (e^H-1)\\
\end{align*}
si deve impostare \(y^H\) in modo che \(u> \widehat{u}\) quindi:
\begin{align*}
  u &= \sqrt{y^H} - (e^H-1) = \widehat{u} = 1\\
  y^H &= \left[ \left( e^H-1 \right)+1 \right]^2\\
  y^H &= 4
\end{align*}

L'impresa si accontenta di uno sforzo \(e^L\) ma paga solo \(y^L\) (si suppone \(e^H=2\), \(e^L=1\))
\begin{align*}
  u &= \sqrt{y^L} - (e^L-1) = \widehat{u} = 1\\
  y^L &= 1
\end{align*}

profitto dell'impresa:
\begin{align*}
  \pi^H &= p^H\pi^G + (1-p^H)\pi^B-y^H = 22\\
  \pi^L &= p^L\pi^G + (1-p^L)\pi^B-y^L = 15\\
\end{align*}

\textbf{Caso di asimmetria informativa}
In questo caso lo sforzo è \(e^L\) a fronte di uno stipendio di \(y^H\):
\begin{align*}
  u &= \sqrt{y^H} - (e^L-1) = 2 > 1\\
  \pi &= p^L\pi^G + (1-p^L)\pi^B-y^H = 12 < 22
\end{align*}

\textbf{Caso di Contratto Incentivante}
\begin{itemize}
  \item \(y^G\) se manager ottiene \(\pi^G\)
  \item \(y^B\) se manager ottiene \(\pi^B\)
\end{itemize}
\begin{equation*}
  u = \left[p^H\sqrt{y^G} + (1-p^H)\sqrt{y^B}\right] - (e^H-1)
\end{equation*}
Vincolo di partecipazione
\begin{equation*}
  u \geq \widehat{u}
\end{equation*}
Vincolo degli incentivi
\begin{equation*}
  u(e^H) \geq  u(e^L))
\end{equation*}
sostituendo si ottiene:
\begin{align*}
  y^G &= 9\\
  y^B &= 0
\end{align*}

Il profitto del manager sará:
\begin{align*}
  y &= p^Hy^G + (1-p^H)y^B = 6\\
  \pi &=  p^H\pi^G + (1-p^H)\pi^B - y = 20
\end{align*}

\subsection{Costi di transazione}

\subsubsection{Caso Alcoa}

\subsubsection{Transazione e Costi di transazione}

\subsubsection{Organizzazione Economica Alternativa}

\subsubsection{Problemi di Hold Up}

\begin{definitions*}
  A & Azienda cliente\\
  B & Azienda Fornitore\\
  C_B & Costi di B\\
  C_A & Costi di A\\
  TVB & Total Variable Costo per numero di pezzi\\
  F & Costo dell'investimento specifico\\
  TVP & Costo variabile di produzione di A (Escluso il prodotto B)\\
  T & Costi addizionali se A non potesse piu usare il prodtto di B (costi di switching)\\
  R & Ricvavi\\
  S & Valore di recupero dell'investimento\\
  switching & se A non può usare il prodotto di B effettua uno switching\\
\end{definitions*}

\begin{align*}
  C_B &= TVB + F\\
  C_A &= TVP + (TVB + F)
\end{align*}

\textbf{Surplus operativo dello scambio}

\begin{align*}
  V &= R-TVB-TVP\\
  \text{Quasi Rendita Appropiabile} = QR &= \text{Surplus (non switching)} - \text{Surplus (switching)}\\
  QR_B &= F-S\\
  QR_A &= V-F-(V-F-T) = T\\
  QR_{Totale} &= QR_A + QR_B = F-S+T\\
  Perdita_B &= Loss_B = F-S-1\\
  Perdita_A &= Loss_A = T-1
\end{align*}

\subsubsection{Contratti}

\textbf{Caso contratto completo - Benchmark con A cliente e B fornitore}, B è sicuro di ricevere il prezzo concordato:

\begin{definitions*}
  c(e) & Costi variabili, dipendono dall'investimento specifico\\
  e & Investimento specifico\\
  p & Prezzo, Ricavi
\end{definitions*}

\begin{align*}
  c_B &= c(e) + e\\
  \pi_B &= \underbrace{p}_{ricavi}-\underbrace{c(e)-e}_{costi}\\
\end{align*}
si massimizza il profitto:
\begin{align*}
  \frac{\partial \pi_B}{\partial e}&= - \frac{\partial c}{\partial e} - 1 = 0\\
  \frac{\partial c}{\partial e} &= -1\\
\end{align*}
quindi un euro in più di investimento genera una riduzione dei costi di un euro (un euro in più di profitto)

\textbf{Caso contratto incompleto}

A riceve il prezzo di p \(\frac{1}{2}\) del margine di contribuzione di B:
\begin{align*}
  mdc_B &= \text(Prezzo di vendita - costi variabili)\\
  p^{holdUp} &= p - \frac{p-c(e)}{2}\\
  \pi^{holdUp}_B &= \frac{p+c(e)}{2} - c(e) - e = \frac{p-c(e)}{2} - e\\
\end{align*}
massimizzo il profitto:
\begin{align*}
  \frac{\partial \pi_B}{\partial e}&= -\frac{1}{2} \frac{\partial c}{\partial e} - 1 = 0\\
  \frac{\partial c}{\partial e}&= - 2
\end{align*}

\subsection{Approccio ai diritti di proprietá}

\subsubsection{Diritti residuali di controllo}

\subsubsection{Separazione Verticale VS Integrazione Verticale}

\subsubsection{Modello di Grossman e Hart}

\begin{definitions*}
  A & Impresa cliente\\
  B & Impresa fornitore\\
  MA & Manager A\\
  MB & Manager di B\\
  e & Investimento specifico del fornitore B (a monte)\\
  i & Investimento specifico del cliente A (a valle)\\
  v & Valore del bene finale senza investimento specifico\\
  VS & separazione verticale\\
  DS & Downstream\\
  US & Upstream\\
  V(1) & Profitto congiunto del primo stadio\\
  V(2) & Profitto congiunto del secondo stadio\\
  s & Valore del bene intermedio senza investimento specifico\\
  p & prezzo\\
  \widehat{p} & prezzo di mercato
\end{definitions*}

\textbf{Caso Benchmark}

\begin{align*}
  i^*&=a^2\\
  e^*&=\alpha^2\\
  V(1)^*&=v-s+a^2+\alpha^2\\
\end{align*}

\textbf{Caso separazione verticale - VS}

\begin{align*}
  i^{VS}&=\frac{(a+c)^2}{4}\\
  e^{VS} &= \frac{(\alpha + \gamma)^2}{4}\\
  V(1)^{VS}&=v-s+\frac{(a+c)(3a-c)}{4}+\frac{(\alpha + \gamma)(3\alpha + \gamma)}{4}\\
  p&=\widehat{p} + (a+c)\sqrt{i} - (\alpha-\gamma)\sqrt{e}
\end{align*}

\textbf{Caso Integrazione a valle - DS}

\begin{align*}
  i^{DS}&=\frac{a^2}{4}\\
  e^{DS} &= \frac{(\alpha + \gamma)^2}{4}\\
  V(1)^{DS}&=v-s+\frac{3a^2}{4}+\frac{(\alpha + \beta)(3\alpha - \beta)}{4}\\
  p&=v + a\sqrt{i} - \alpha\sqrt{e} + \beta\sqrt{e}
\end{align*}

\textbf{Caso Integrazione a monte - US}

\begin{align*}
  i^{US}&=\frac{(a+b)^2}{4}\\
  e^{US} &= \frac{\alpha^2}{4}\\
  V(1)^{US}&=v-s+\frac{3\alpha^2}{4}+\frac{(a + b)(3a - b)}{4}\\
  p&=v + a\sqrt{i} - \alpha\sqrt{e} + \beta\sqrt{e}
\end{align*}

Investimento \(e\) del fornitore \(B\): Benchamark > DS > VS > US

Investimento \(i\) del fornitore \(A\): Benchamark > US > VS > DS

\section{TEORIE FINANZIARIE}

\subsection{Struttura Finanziaria}

\subsubsection{Corporate governance e struttura finanziaria}

\subsubsection{conflitti di interesse}

\subsubsection{Metodi di finanziamento delle imprese}

\subsubsection{Modello di Modigliani Miller}

\begin{definitions*}
  V & Valore dell'impresa\\
  B & Debito dell'impresa \\
  r/times B & oneri finanziari\\
  p & Prezzo (valore) del diritto al profitto per l'azionista\\
  r_{wacc} & Costo del capitale\\
  r_s & Costo dell'equity / rendimento dell'equity (Caso levered) / ROE\\
  r_0 & Costo dell'equity / rendimento dell'equity (Caso unlevered) / ROA\\
  r_B & Tasso di interesse del debito / costo del debito\\
  B & Valore del debito\\
  S & valore dell'equity\\
  T_c & Aliquota fiscale\\
  EPS & Earning per Share (Guadagno per azione)
\end{definitions*}

\begin{align*}
  V_U &= EBIT(1-T_c) = \underbrace{p(x)}_{equity}\\
  V_L &= EBIT(1-T_c) + r_B\times B\times T_c = V_U + r_B\times B\times T_c = \underbrace{B}_{debito}+\underbrace{p(x-B(1+r))}_{equity}\\
  ROE &= \frac{\text{Net Income}}{\text{Equity}}\\
  ROA &= \frac{EBIT}{Assets}\\
  EPS &= \frac{\text{Net Income}}{\text{Numero azioni}}\\
\end{align*}
c'è indifferenza tra la strategia equity-levered se \(V_L = V_U\):
\begin{align*}
  p(x) &= B + p(x-B(1+r))
\end{align*}

\textbf{Caso senza tasse e senza costi di bancarotta}
\begin{figure}[H]
  \centering
  \includegraphics[width=.4\linewidth]{images/1.jpg}
\end{figure}

\begin{align*}
  V_L &= V_U\\
  r_s &= r_0 + \frac{B}{S}\left(r_0-R_B\right)\\
  r_{wacc} &= \frac{B}{B+S}r_B + \frac{S}{B+S}r_s
\end{align*}

\textbf{Caso con tasse e senza costi di bancarotta}
\begin{figure}[H]
  \centering
  \includegraphics[width=.4\linewidth]{images/2.jpg}
\end{figure}

\begin{align*}
  V_L &= V_U + T_cB\\
  r_s &= r_0 + \frac{B}{S}\left(1-T_c\right)\left(r_0-R_B\right)\\
  r_{wacc} &= \frac{B}{B+S}r_B(1-T_c) + \frac{S}{B+S}r_s
\end{align*}

\textbf{Caso con tasse e con costi di bancarotta}
\begin{figure}[H]
  \centering
  \includegraphics[width=.4\linewidth]{images/3.jpg}
\end{figure}

\subsection{Struttura Proprietaria}

\subsubsection{Modello di Jensen e Meckling}

\begin{definitions*}
  BM & Benefici Monetari\\
  BNM & Benefici Non Monetari\\
  \alpha & Quota venduta dall'imprenditore agli azionisti esterni\\
  E & Imprenditore\\
  e & sforzo dell'imprenditore\\
  V & Valore dell'impresa\\
  c & Costo dello sforzo\\
  v \geq 0; v'>0; v''<0 & Valore marginalmente decrescente\\
  c \geq 0; c'>0; c''>0 & Costo marginalmente crescente\\
\end{definitions*}

\textbf{Caso in cui E possiede il 100\% dell'impresa - Benchmark}

\begin{align*}
  MAX U_m(e)&=MAX[v(e)-c(e)]\\
  \frac{\partial U_m(e)}{\partial e} &= \frac{\partial V(e)}{\partial e} - \frac{\partial c(e)}{\partial e}\\
  \frac{\partial v(e)}{\partial e} &= \frac{\partial c(e)}{\partial e}\\
  e^* &= \text{livello di sforzo ottimale}\\
\end{align*}

\textbf{Caso in cui E vende una quota \(\alpha\) della sua impresa}

\begin{align*}
  MAX U_m(e)&=MAX[(1-\alpha)v(e)-c(e)]\\
  \frac{\partial U_m(e)}{\partial e} &= \frac{\partial [(1-\alpha)V(e)]}{\partial e} - \frac{\partial c(e)}{\partial e}\\
  \frac{\partial [(1-\alpha)v(e)]}{\partial e} &= \frac{\partial c(e)}{\partial e}\\
  \widehat{e} &= \text{livello di sforzo efficiente} < e^*\\
\end{align*}

\textbf{Caso in cui E seleziona il progetto - selezione avversa}

\begin{align*}
  \text{underinvestment} \rightarrow \text{ineefficienza ex-ante}\\
  \text{underpricing} \rightarrow \text{ineefficienza ex-post}
\end{align*}

\subsubsection{Costi di agenzia del capitale}

\subsubsection{Modello di Jensen}

\begin{definitions*}
  D & Obbligazioni\\
  r & Tasso di interesse\\
  E & Imprenditore
\end{definitions*}

\begin{align*}
  MAX U_m(e) &= MAX[v(e) - c(e) e D(1+r)]\\
  \frac{\partial U_m(e)}{\partial e} &= \frac{\partial v(e)}{\partial e} - \frac{\partial c(e)}{\partial e} = 0\\
  \frac{\partial v(e)}{\partial e}& = \frac{\partial c(e)}{\partial e}\\
  e^* &= \text{Livello di sforzo efficiente}
\end{align*}

\subsubsection{Struttura proprietaria ottimale}

\section{CORPORATE GOVERNANCE}

\section{GOVERNANCE}

\end{document}